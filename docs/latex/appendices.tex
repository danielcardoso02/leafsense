\chapter{System Configuration Reference}

\setlength{\parindent}{0pt}
\hspace{1cm}{This appendix provides quick-reference tables for the LeafSense hardware and software specifications.}

\section{Hardware Platform}

\begin{table}[H]
    \begin{center}
        \caption{LeafSense Hardware Specifications}
        \begin{tabular}{|c|c|}
            \hline
            \textbf{Component} & \textbf{Specification} \\
            \hline
            Development Board & Raspberry Pi 4 Model B (2GB RAM) \\
            \hline
            Processor & BCM2711, Quad-core Cortex-A72 @ 1.8GHz \\
            \hline
            Operating System & Custom Buildroot 2025.08 Linux \\
            \hline
            Kernel Version & 6.12.41-v8 (ARM64) \\
            \hline
            Storage & 32GB microSD card \\
            \hline
            Display & Waveshare 3.5" ILI9486 LCD (480$\times$320) \\
            \hline
            Touchscreen & ADS7846 resistive touch controller \\
            \hline
            Camera & OV5647 5MP module (/dev/video0) \\
            \hline
        \end{tabular}
        \label{tab:hw-specs}
    \end{center}
\end{table}

\section{Sensor Configuration}

\begin{table}[H]
    \begin{center}
        \caption{LeafSense Sensor Specifications}
        \begin{tabular}{|c|c|c|c|}
            \hline
            \textbf{Sensor} & \textbf{Model} & \textbf{Interface} & \textbf{Range} \\
            \hline
            Temperature & DS18B20 & 1-Wire (GPIO 19) & -10 to 85°C \\
            \hline
            pH Sensor & PH-4502C & Analog (ADS1115 CH0) & pH 0--14 \\
            \hline
            TDS Sensor & Seeed Grove & Analog (ADS1115 CH1) & 0--1000 ppm \\
            \hline
            ADC Module & ADS1115 16-bit & I2C (0x48) & 4 channels \\
            \hline
            RTC Module & DS3231 & I2C (0x68) & Battery backup \\
            \hline
        \end{tabular}
        \label{tab:sensor-specs}
    \end{center}
\end{table}

\section{Actuator Configuration}

\begin{table}[H]
    \begin{center}
        \caption{LeafSense Actuator Specifications}
        \begin{tabular}{|c|c|c|}
            \hline
            \textbf{Actuator} & \textbf{Specification} & \textbf{Control} \\
            \hline
            Alert LED & GPIO 20 & Custom kernel module (/dev/led0) \\
            \hline
            Water Heater & 220V/200W submersible & Relay (GPIO 26) \\
            \hline
            pH Up Pump & Peristaltic 3.3V & GPIO 6 \\
            \hline
            pH Down Pump & Peristaltic 3.3V & GPIO 13 \\
            \hline
            Nutrient Pump & Peristaltic 3.3V & GPIO 5 \\
            \hline
        \end{tabular}
        \label{tab:actuator-specs}
    \end{center}
\end{table}

\section{Software Stack}

\begin{table}[H]
    \begin{center}
        \caption{LeafSense Software Components}
        \begin{tabular}{|c|c|c|}
            \hline
            \textbf{Component} & \textbf{Version} & \textbf{Purpose} \\
            \hline
            Qt5 & 5.15.x & GUI framework (linuxfb plugin) \\
            \hline
            OpenCV & 4.x & Computer vision and image capture \\
            \hline
            ONNX Runtime & 1.20.0 & Machine learning inference (C++ API) \\
            \hline
            SQLite3 & 3.x & Local database (/opt/leafsense/leafsense.db) \\
            \hline
            libgpiod & 1.6.x & GPIO control library \\
            \hline
            Dropbear & 2022.x & SSH server for remote access \\
            \hline
            GCC & 14.3.0 & Cross-compiler (ARM64) \\
            \hline
        \end{tabular}
        \label{tab:sw-specs}
    \end{center}
\end{table}

\chapter{GPIO Pin Assignments}

\setlength{\parindent}{0pt}
\hspace{1cm}{This appendix documents the complete GPIO pin mapping for the LeafSense system.}

\begin{table}[H]
    \begin{center}
        \caption{Raspberry Pi GPIO Pin Connections}
        \begin{tabular}{|c|c|c|c|}
            \hline
            \textbf{GPIO} & \textbf{Function} & \textbf{Device} & \textbf{Notes} \\
            \hline
            2 & SDA (I2C Data) & ADC, RTC & I2C bus 1 \\
            \hline
            3 & SCL (I2C Clock) & ADC, RTC & I2C bus 1 \\
            \hline
            5 & Digital Output & Nutrient Pump & Active high \\
            \hline
            6 & Digital Output & pH Up Pump & Active high \\
            \hline
            7 & SPI CE1 & Display & Chip enable \\
            \hline
            8 & SPI CE0 & Display & Chip enable \\
            \hline
            9 & SPI MISO & Display & Data in \\
            \hline
            10 & SPI MOSI & Display & Data out \\
            \hline
            11 & SPI CLK & Display & Clock \\
            \hline
            13 & Digital Output & pH Down Pump & Active high \\
            \hline
            17 & Digital Output & Display DC & Data/Command \\
            \hline
            19 & 1-Wire Data & DS18B20 & w1-gpio overlay \\
            \hline
            20 & Digital Output & Alert LED & Kernel module \\
            \hline
            24 & Digital Output & Display RST & Reset \\
            \hline
            25 & Touch IRQ & ADS7846 & Touch interrupt \\
            \hline
            26 & Digital Output & Relay Control & Heater relay \\
            \hline
        \end{tabular}
        \label{tab:gpio-pins}
    \end{center}
\end{table}

\section{I2C Device Addresses}

\begin{table}[H]
    \begin{center}
        \caption{I2C Device Configuration}
        \begin{tabular}{|c|c|c|}
            \hline
            \textbf{Address} & \textbf{Device} & \textbf{Configuration} \\
            \hline
            0x48 & ADS1115 ADC & 16-bit, 860 SPS, 4 channels \\
            \hline
            0x68 & DS3231 RTC & Battery-backed real-time clock \\
            \hline
        \end{tabular}
        \label{tab:i2c-devices}
    \end{center}
\end{table}

\chapter{Troubleshooting Guide}

\setlength{\parindent}{0pt}
\hspace{1cm}{This appendix provides solutions to common problems encountered during LeafSense deployment and operation.}

\section{Compilation Issues}

\subsection{Qt5 Not Found}

\begin{lstlisting}[language=bash, caption={Installing Qt5 dependencies}]
# Ubuntu/Debian host
sudo apt install qt5-default qtcharts5-dev libqt5svg5-dev libqt5sql5-sqlite

# Specify Qt5 path explicitly
cmake -DQt5_DIR=/path/to/qt5/lib/cmake/Qt5 ..
\end{lstlisting}

\subsection{OpenCV Not Found}

\begin{lstlisting}[language=bash, caption={Installing OpenCV}]
# Ubuntu/Debian host
sudo apt install libopencv-dev

# Verify installation
pkg-config --modversion opencv4
\end{lstlisting}

\subsection{C++17 Filesystem Error}

\setlength{\parindent}{0pt}
\hspace{1cm}{Add to CMakeLists.txt:}

\begin{lstlisting}[language=cmake, caption={CMake C++17 configuration}]
set(CMAKE_CXX_STANDARD 17)
set(CMAKE_CXX_STANDARD_REQUIRED ON)
target_link_libraries(LeafSense PRIVATE stdc++fs)
\end{lstlisting}

\subsection{ioremap\_nocache Error (Kernel 5.6+)}

\setlength{\parindent}{0pt}
\hspace{1cm}{In kernel module code, replace deprecated function:}

\begin{lstlisting}[language=c, caption={Fix for ioremap\_nocache}]
// Before (deprecated in Linux 5.6+)
gpio_base = ioremap_nocache(GPIO_BASE, GPIO_SIZE);

// After (correct for modern kernels)
gpio_base = ioremap(GPIO_BASE, GPIO_SIZE);
\end{lstlisting}

\section{Runtime Issues}

\subsection{Library Not Found on Target}

\begin{lstlisting}[language=bash, caption={Copying missing Qt libraries}]
# Check available libraries on Raspberry Pi
ssh root@10.42.0.196 "ls /usr/lib/libQt5*"

# Copy missing library from Buildroot output
scp ~/buildroot/output/target/usr/lib/libQt5Charts.so.5.15.14 \
    root@10.42.0.196:/usr/lib/

# Create required symlinks
ssh root@10.42.0.196 "cd /usr/lib && \
    ln -sf libQt5Charts.so.5.15.14 libQt5Charts.so.5 && \
    ln -sf libQt5Charts.so.5.15.14 libQt5Charts.so"
\end{lstlisting}

\subsection{Qt Platform Plugin Not Found}

\begin{lstlisting}[language=bash, caption={Qt platform configuration}]
# For Waveshare 3.5" LCD (framebuffer 1)
export QT_QPA_PLATFORM=linuxfb:fb=/dev/fb1:size=480x320

# For HDMI output (framebuffer 0)
export QT_QPA_PLATFORM=linuxfb:fb=/dev/fb0

# For headless testing
export QT_QPA_PLATFORM=offscreen
\end{lstlisting}

\subsection{Permission Denied for /dev/led0}

\begin{lstlisting}[language=bash, caption={LED device permissions}]
# Option 1: Run application as root
sudo ./LeafSense

# Option 2: Fix device permissions
chmod 666 /dev/led0

# Option 3: Add udev rule for persistent permissions
echo 'KERNEL=="led0", MODE="0666"' > /etc/udev/rules.d/99-led.rules
udevadm control --reload-rules
\end{lstlisting}

\section{Hardware Issues}

\subsection{Camera Not Detected}

\begin{lstlisting}[language=bash, caption={Camera troubleshooting}]
# Check if camera device exists
ls -la /dev/video*

# Test camera with v4l2-ctl
v4l2-ctl --list-devices
v4l2-ctl -d /dev/video0 --all

# Verify camera is enabled in config.txt
cat /boot/config.txt | grep -E "start_x|gpu_mem|camera"

# Required settings in /boot/config.txt:
# start_x=1 and # gpu_mem=128
\end{lstlisting}

\subsection{Touchscreen Not Responding}

\begin{lstlisting}[language=bash, caption={Touchscreen troubleshooting}]
# Check input devices
cat /proc/bus/input/devices | grep -A5 ads7846

# Verify SPI is enabled
ls /dev/spidev*

# Check for ads7846 driver in kernel log
dmesg | grep -i ads7846

# Test touch events
cat /dev/input/event0  # (raw touch data)
\end{lstlisting}

\subsection{Temperature Sensor Not Reading}

\begin{lstlisting}[language=bash, caption={DS18B20 1-Wire troubleshooting}]
# Check 1-Wire devices are detected
ls /sys/bus/w1/devices/

# Expected output: 28-xxxxxxxxxxxx (where xx is device ID)

# Read raw temperature (millidegrees)
cat /sys/bus/w1/devices/28-*/temperature

# Verify w1-gpio overlay is enabled
dtoverlay -l | grep w1

# Required in /boot/config.txt:
# dtoverlay=w1-gpio,gpiopin=19
\end{lstlisting}

\subsection{I2C Devices Not Detected}

\begin{lstlisting}[language=bash, caption={I2C troubleshooting}]
# Scan I2C bus for devices
i2cdetect -y 1

# Expected addresses:
# 0x48 - ADS1115 ADC and # 0x68 - DS3231 RTC

# Check I2C kernel module
lsmod | grep i2c

# Verify I2C is enabled
cat /boot/config.txt | grep dtparam=i2c
\end{lstlisting}

\section{Database Issues}

\subsection{Database Locked}

\begin{lstlisting}[language=bash, caption={Fixing SQLite database lock}]
# Identify process holding the lock
fuser /opt/leafsense/leafsense.db

# Kill blocking process if necessary
kill -9 <PID>

# Verify database integrity
sqlite3 /opt/leafsense/leafsense.db "PRAGMA integrity_check;"

# If corrupted, recover from backup or reinitialize
sqlite3 /opt/leafsense/leafsense.db < /opt/leafsense/schema.sql
\end{lstlisting}

\section{Machine Learning Issues}

\subsection{ONNX Model Not Loading}

\begin{lstlisting}[language=bash, caption={ML model troubleshooting}]
# Verify model file exists and has correct permissions
ls -la /opt/leafsense/leafsense_model.onnx

# Check ONNX Runtime library is available
ls -la /usr/lib/libonnxruntime.so*

# If library is missing, copy from external/
scp external/onnxruntime-arm64/lib/libonnxruntime.so* \
    root@10.42.0.196:/usr/lib/

# Create symlinks if needed
ssh root@10.42.0.196 "cd /usr/lib && \
    ln -sf libonnxruntime.so.1.16.3 libonnxruntime.so"
\end{lstlisting}

\subsection{Out-of-Distribution (OOD) Detection}

\setlength{\parindent}{0pt}
\hspace{1cm}{The system uses combined OOD detection (entropy + green ratio) to reject non-plant images. This two-stage approach prevents the model from confidently misclassifying non-plant objects:}

\begin{lstlisting}[language=bash, caption={Understanding OOD detection output}]
# Normal plant prediction (valid):
[ML] Green ratio: 9.63% (passed)
[ML] Prediction: Pest Damage (confidence: 99.1%, entropy: 0.12, valid: yes)

# Non-plant image rejected (OOD via green ratio):
[ML] Green ratio: 4.64% (failed, threshold: 5.00%)
[ML] Prediction: Unknown (Not a Plant) (confidence: 89%, entropy: 0.52, valid: no)

# High entropy rejection:
[ML] High entropy (1.85 > 1.8) - possible non-plant image
[ML] Prediction: Unknown (Not a Plant) (confidence: 35%, entropy: 1.85, valid: no)
\end{lstlisting}

\begin{table}[H]
    \begin{center}
        \caption{OOD Detection Thresholds}
        \begin{tabular}{|c|c|c|}
            \hline
            \textbf{Threshold} & \textbf{Value} & \textbf{Effect} \\
            \hline
            ENTROPY\_THRESHOLD & 1.8 & Reject if entropy exceeds (max=2.0 for 4 classes) \\
            \hline
            MIN\_CONFIDENCE & 30\% & Reject if top class confidence below \\
            \hline
            MIN\_GREEN\_RATIO & 10\% & Reject if green pixels below (HSV, tuned for lettuce) \\
            \hline
        \end{tabular}
        \label{tab:ood-thresholds}
    \end{center}
\end{table}

\hspace{1cm}{The green ratio check uses HSV color space to detect plant-like pixels (green hue 35-85°, yellow-green 20-35°). This prevents objects like keyboards, shoes, or other non-green items from being classified even when the model is confident.}

\subsection{ML Prediction Always Returns Same Class}

\begin{lstlisting}[language=bash, caption={Debugging ML predictions}]
# Check if model is loading (look for mock mode message)
# If you see this, the model file is not found:
[ML] Warning: Model file not found: /opt/leafsense/leafsense_model.onnx
[ML] Running in mock mode (always returns Healthy)

# Verify classes file exists
cat /opt/leafsense/leafsense_model_classes.txt
# Expected output:
# deficiency
# disease
# healthy
# pest
\end{lstlisting}

\section{Diagnostic Commands}

\begin{lstlisting}[language=bash, caption={System diagnostic commands}]
# Check LeafSense process status
ps aux | grep -i leafsense

# View kernel messages (driver loading, errors)
dmesg | tail -50

# Check disk space
df -h /opt/leafsense

# Monitor system resources
top -b -n 1 | head -20

# Check network connectivity
ip addr show usb0
ping -c 3 10.42.0.1

# View application logs
cat /var/log/messages | grep -i leafsense

# Check LED module status
cat /sys/module/led/parameters/* 2>/dev/null || echo "Module not loaded"
lsmod | grep led
\end{lstlisting}

\section{Quick Reference: Common Commands}

\begin{table}[H]
    \begin{center}
        \caption{Frequently Used Commands}
        \begin{tabular}{|c|c|}
            \hline
            \textbf{Task} & \textbf{Command} \\
            \hline
            Start application & \texttt{cd /opt/leafsense \&\& ./LeafSense} \\
            \hline
            Load LED module & \texttt{insmod /root/led.ko} \\
            \hline
            Unload LED module & \texttt{rmmod led} \\
            \hline
            Test LED & \texttt{echo 1 > /dev/led0} \\
            \hline
            Check temperature & \texttt{cat /sys/bus/w1/devices/28-*/temperature} \\
            \hline
            Scan I2C bus & \texttt{i2cdetect -y 1} \\
            \hline
            View database & \texttt{sqlite3 /opt/leafsense/leafsense.db} \\
            \hline
            Check camera & \texttt{v4l2-ctl -d /dev/video0 --all} \\
            \hline
            Reboot system & \texttt{reboot} \\
            \hline
        \end{tabular}
        \label{tab:quick-commands}
    \end{center}
\end{table}